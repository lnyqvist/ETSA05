
%% report_template.tex
%% V1.0
%% 2012-03-16
%% by Jesper Pedersen Notander
%% See:
%% http://www.cs.lth.se/jesper_pedersen_notander
%% for current contact information.
%%
%% This is a template file contaning instructions and a skeleton outline 
%% for the final report in the course ETSA05: Software Engineering 
%% Process - Soft Issues, given by the Department of Computer Science at 
%% Lund University, Sweden.
%% 
%% This template requires IEEEtran.cls, written by Michael Shell, version 
%% 1.7 or later.
%%
%% Support sites:
%% http://www.cs.lth.se/etsa05/
%% http://www.ieee.org/

%%*************************************************************************
%% Legal Notice:
%% This code is offered as-is without any warranty either expressed or
%% implied; without even the implied warranty of MERCHANTABILITY or
%% FITNESS FOR A PARTICULAR PURPOSE! 
%%
%% User assumes all risk.
%%
%% In no event shall Lund University or any contributor to this code be 
%% liable for any damages or losses, including, but not limited to, 
%% incidental, consequential, or any other damages, resulting from the use 
%% or misuse of any information contained here.
%%
%% All comments are the opinions of their respective authors and are not
%% necessarily endorsed by Lund University.
%%
%% This work is distributed under the LaTeX Project Public License (LPPL)
%% ( http://www.latex-project.org/ ) version 1.3, and may be freely used,
%% distributed and modified. A copy of the LPPL, version 1.3, is included
%% in the base LaTeX documentation of all distributions of LaTeX released
%% 2003/12/01 or later.
%%
%% Retain all contribution notices and credits.
%% ** Modified files should be clearly indicated as such, including  **
%% ** renaming them and changing author support contact information. **
%%
%% File list of work: report_template.tex
%%*************************************************************************


\documentclass[conference]{IEEEtran}
% If IEEEtran.cls has not been installed into the LaTeX system files,
% manually specify the path to it like:
% \documentclass[conference]{../sty/IEEEtran}

\begin{document}

\title{Title}


% author names and affiliations
% use a multiple column layout for up to three different
% affiliations
\author{\IEEEauthorblockN{Names/s per 1st Affiliation (Author)}
\IEEEauthorblockA{line 1 (of Affiliation): dept. name of organization\\
line 2: name of organization, acronyms acceptable\\
line 3: City, Country\\
line 4: e-mailaddress if desired}
\and
\IEEEauthorblockN{Jesper Pedersen Notander \\and Richard Berntsson Svensson\\}
\IEEEauthorblockA{Software Engineering Research Group\\
Dept. of Computer Science, Lund University\\
Lund, Sweden\\
http://www.cs.lth.se/ets05}}


\maketitle


\begin{abstract}
This document is a template. The various components of your paper (title, headings, etc.) are already defined on the style sheet, as illustrated by the portions given in this document.
\end{abstract}

\section{Introduction}
The introduction section can be used to introduce the company in general or to introduce the purpose and context of this report. This template is a document that provides the predefined outline of your group essay from the seminar compendium. If you want to change or adjust your outline, you must do so before the outline is due (March 30, 2012). It is important to not that you have a page limit of 5-7 pages for your final report (which is due May 11, 2012), hence it is important to decide how much should be analyzed, discussed, and written for each section.

\section{Description of the System}
In this section you should describe the system that your group has selected. Your system description should not only describe the technical parts of the system, but also from a user point of view. 

\section{Quality Characteristics}
Under this section you should discuss which quality characteristics are particular important for this system. Which quality characteristics are considered most important, i.e. prioritize among the relevant quality characteristics, and provide a rationale for the prioritization. In addition, from which perspective is the prioritization important. You find the quality characteristics in the ISO 9126 standard in one article, e.g. \cite{jung2004} in the articles course compendium. The prioritization can be performed at the characteristics level and/or at the sub-characteristics level.

\section{Availability for Disabled}
Here you should provide a discussion/analysis of which aspects of availability for the elderly, disabled, and individuals with special needs are current interests. Analyze how can Information and Communication Technology (ICT) help people with disability. However, if your system is not applicable for people with disabilities, analyze the system with respect to accessibility with respect to general human-computer interaction. 

\section{Ethical Aspects}
Which ethical questions must be answered for the system? Here you should identify, and discuss, ethical dilemmas/potential ethical issues related to your system. Examples of different kinds of ethical dilemmas can be found in Berenbach and Broy \cite{berenbach2004}.

\section{Legal Aspects}
In this section you should discuss and analyze which legal aspects, e.g. intellectual property, are relevant for your system.

\section{Financial Aspects}
Under this section financial aspects should be analyzed and discussed. What are the financial driving forces in favor of and against your system? Who are the investors, who can make a profit, what is the cost for the individual, and what is the cost for society? Which business model is used for your system, and what investment strategy is used (short-term or long-term)? These questions are examples of what can be analyzed with regards to financial aspects. 

\section{Summary}
In the summary section you should summarize what your report is about and present your main findings/consequences. 


% references section
\begin{thebibliography}{1}
\bibitem{jung2004}
H-W. Jung, S-G Kim, and C-S Chung, ''Measuring Software Product Quality: A Survey of ISO/IEC 9126'' IEEE Software, 21(5), pp. 88–92, 2004.

\bibitem{berenbach2004}
B. Berenbach, and M. Broy, ''Professional and ethical dilemmas in software engineering'' Computer, 42(1), pp. 74–88, 2004

\end{thebibliography}

\end{document}


